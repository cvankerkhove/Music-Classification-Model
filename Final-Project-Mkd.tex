% Options for packages loaded elsewhere
\PassOptionsToPackage{unicode}{hyperref}
\PassOptionsToPackage{hyphens}{url}
%
\documentclass[
]{article}
\usepackage{lmodern}
\usepackage{amssymb,amsmath}
\usepackage{ifxetex,ifluatex}
\ifnum 0\ifxetex 1\fi\ifluatex 1\fi=0 % if pdftex
  \usepackage[T1]{fontenc}
  \usepackage[utf8]{inputenc}
  \usepackage{textcomp} % provide euro and other symbols
\else % if luatex or xetex
  \usepackage{unicode-math}
  \defaultfontfeatures{Scale=MatchLowercase}
  \defaultfontfeatures[\rmfamily]{Ligatures=TeX,Scale=1}
\fi
% Use upquote if available, for straight quotes in verbatim environments
\IfFileExists{upquote.sty}{\usepackage{upquote}}{}
\IfFileExists{microtype.sty}{% use microtype if available
  \usepackage[]{microtype}
  \UseMicrotypeSet[protrusion]{basicmath} % disable protrusion for tt fonts
}{}
\makeatletter
\@ifundefined{KOMAClassName}{% if non-KOMA class
  \IfFileExists{parskip.sty}{%
    \usepackage{parskip}
  }{% else
    \setlength{\parindent}{0pt}
    \setlength{\parskip}{6pt plus 2pt minus 1pt}}
}{% if KOMA class
  \KOMAoptions{parskip=half}}
\makeatother
\usepackage{xcolor}
\IfFileExists{xurl.sty}{\usepackage{xurl}}{} % add URL line breaks if available
\IfFileExists{bookmark.sty}{\usepackage{bookmark}}{\usepackage{hyperref}}
\hypersetup{
  pdftitle={4740 Project},
  pdfauthor={Chris VanKerkhove, John Bush},
  hidelinks,
  pdfcreator={LaTeX via pandoc}}
\urlstyle{same} % disable monospaced font for URLs
\usepackage[margin=1in]{geometry}
\usepackage{color}
\usepackage{fancyvrb}
\newcommand{\VerbBar}{|}
\newcommand{\VERB}{\Verb[commandchars=\\\{\}]}
\DefineVerbatimEnvironment{Highlighting}{Verbatim}{commandchars=\\\{\}}
% Add ',fontsize=\small' for more characters per line
\usepackage{framed}
\definecolor{shadecolor}{RGB}{248,248,248}
\newenvironment{Shaded}{\begin{snugshade}}{\end{snugshade}}
\newcommand{\AlertTok}[1]{\textcolor[rgb]{0.94,0.16,0.16}{#1}}
\newcommand{\AnnotationTok}[1]{\textcolor[rgb]{0.56,0.35,0.01}{\textbf{\textit{#1}}}}
\newcommand{\AttributeTok}[1]{\textcolor[rgb]{0.77,0.63,0.00}{#1}}
\newcommand{\BaseNTok}[1]{\textcolor[rgb]{0.00,0.00,0.81}{#1}}
\newcommand{\BuiltInTok}[1]{#1}
\newcommand{\CharTok}[1]{\textcolor[rgb]{0.31,0.60,0.02}{#1}}
\newcommand{\CommentTok}[1]{\textcolor[rgb]{0.56,0.35,0.01}{\textit{#1}}}
\newcommand{\CommentVarTok}[1]{\textcolor[rgb]{0.56,0.35,0.01}{\textbf{\textit{#1}}}}
\newcommand{\ConstantTok}[1]{\textcolor[rgb]{0.00,0.00,0.00}{#1}}
\newcommand{\ControlFlowTok}[1]{\textcolor[rgb]{0.13,0.29,0.53}{\textbf{#1}}}
\newcommand{\DataTypeTok}[1]{\textcolor[rgb]{0.13,0.29,0.53}{#1}}
\newcommand{\DecValTok}[1]{\textcolor[rgb]{0.00,0.00,0.81}{#1}}
\newcommand{\DocumentationTok}[1]{\textcolor[rgb]{0.56,0.35,0.01}{\textbf{\textit{#1}}}}
\newcommand{\ErrorTok}[1]{\textcolor[rgb]{0.64,0.00,0.00}{\textbf{#1}}}
\newcommand{\ExtensionTok}[1]{#1}
\newcommand{\FloatTok}[1]{\textcolor[rgb]{0.00,0.00,0.81}{#1}}
\newcommand{\FunctionTok}[1]{\textcolor[rgb]{0.00,0.00,0.00}{#1}}
\newcommand{\ImportTok}[1]{#1}
\newcommand{\InformationTok}[1]{\textcolor[rgb]{0.56,0.35,0.01}{\textbf{\textit{#1}}}}
\newcommand{\KeywordTok}[1]{\textcolor[rgb]{0.13,0.29,0.53}{\textbf{#1}}}
\newcommand{\NormalTok}[1]{#1}
\newcommand{\OperatorTok}[1]{\textcolor[rgb]{0.81,0.36,0.00}{\textbf{#1}}}
\newcommand{\OtherTok}[1]{\textcolor[rgb]{0.56,0.35,0.01}{#1}}
\newcommand{\PreprocessorTok}[1]{\textcolor[rgb]{0.56,0.35,0.01}{\textit{#1}}}
\newcommand{\RegionMarkerTok}[1]{#1}
\newcommand{\SpecialCharTok}[1]{\textcolor[rgb]{0.00,0.00,0.00}{#1}}
\newcommand{\SpecialStringTok}[1]{\textcolor[rgb]{0.31,0.60,0.02}{#1}}
\newcommand{\StringTok}[1]{\textcolor[rgb]{0.31,0.60,0.02}{#1}}
\newcommand{\VariableTok}[1]{\textcolor[rgb]{0.00,0.00,0.00}{#1}}
\newcommand{\VerbatimStringTok}[1]{\textcolor[rgb]{0.31,0.60,0.02}{#1}}
\newcommand{\WarningTok}[1]{\textcolor[rgb]{0.56,0.35,0.01}{\textbf{\textit{#1}}}}
\usepackage{graphicx,grffile}
\makeatletter
\def\maxwidth{\ifdim\Gin@nat@width>\linewidth\linewidth\else\Gin@nat@width\fi}
\def\maxheight{\ifdim\Gin@nat@height>\textheight\textheight\else\Gin@nat@height\fi}
\makeatother
% Scale images if necessary, so that they will not overflow the page
% margins by default, and it is still possible to overwrite the defaults
% using explicit options in \includegraphics[width, height, ...]{}
\setkeys{Gin}{width=\maxwidth,height=\maxheight,keepaspectratio}
% Set default figure placement to htbp
\makeatletter
\def\fps@figure{htbp}
\makeatother
\setlength{\emergencystretch}{3em} % prevent overfull lines
\providecommand{\tightlist}{%
  \setlength{\itemsep}{0pt}\setlength{\parskip}{0pt}}
\setcounter{secnumdepth}{-\maxdimen} % remove section numbering

\title{4740 Project}
\author{Chris VanKerkhove, John Bush}
\date{5/25/2021}

\begin{document}
\maketitle

\hypertarget{a-music-assignment-model}{%
\subsection{A Music Assignment Model}\label{a-music-assignment-model}}

In this Markdown, I will document my process of creating and validating
different models (with increasing complexity) for classifying music
snippets into different musical genres.

The data is taken from Kaggle from the page
\url{https://www.kaggle.com/insiyeah/musicfeatures}. The data was
generated using 30 second MP3 snippets of songs and then extracting
features of the sound-bite to act as different predictors.

\hypertarget{data-preproccessing}{%
\subsubsection{Data Preproccessing}\label{data-preproccessing}}

Checking for missing values, casting types for classification, ensuring
clean data.frame

\begin{Shaded}
\begin{Highlighting}[]
\CommentTok{#Dataset is complete, No missing data}
\KeywordTok{sum}\NormalTok{(missing)}
\end{Highlighting}
\end{Shaded}

\begin{verbatim}
## [1] 0
\end{verbatim}

\begin{Shaded}
\begin{Highlighting}[]
\CommentTok{#removing file name column ('label' column acts as dependent var)}
\NormalTok{df <-}\StringTok{ }\KeywordTok{select}\NormalTok{(music.dat, }\OperatorTok{-}\NormalTok{filename)}
\NormalTok{df}\OperatorTok{$}\NormalTok{label <-}\StringTok{ }\KeywordTok{as.factor}\NormalTok{(df}\OperatorTok{$}\NormalTok{label)}
\KeywordTok{tibble}\NormalTok{(df)}
\end{Highlighting}
\end{Shaded}

\begin{verbatim}
## Registered S3 method overwritten by 'cli':
##   method     from
##   print.tree tree
\end{verbatim}

\begin{verbatim}
## # A tibble: 1,000 x 29
##    tempo beats chroma_stft   rmse spectral_centro~ spectral_bandwi~ rolloff
##    <dbl> <dbl>       <dbl>  <dbl>            <dbl>            <dbl>   <dbl>
##  1 103.     50       0.380 0.248             2117.            1957.   4196.
##  2  95.7    44       0.306 0.113             1156.            1498.   2170.
##  3 152.     75       0.253 0.152             1331.            1974.   2900.
##  4 185.     91       0.269 0.119             1361.            1568.   2740.
##  5 161.     74       0.391 0.138             1811.            2052.   3928.
##  6 108.     51       0.357 0.162             2068.            2034.   4231.
##  7 161.     80       0.375 0.110             2340.            2257.   4973.
##  8 152.     74       0.431 0.196             1947.            1980.   3956.
##  9  92.3    45       0.291 0.0892            1110.            1463.   2244.
## 10 152.     70       0.329 0.0670            1172.            1706.   2345.
## # ... with 990 more rows, and 22 more variables: zero_crossing_rate <dbl>,
## #   mfcc1 <dbl>, mfcc2 <dbl>, mfcc3 <dbl>, mfcc4 <dbl>, mfcc5 <dbl>,
## #   mfcc6 <dbl>, mfcc7 <dbl>, mfcc8 <dbl>, mfcc9 <dbl>, mfcc10 <dbl>,
## #   mfcc11 <dbl>, mfcc12 <dbl>, mfcc13 <dbl>, mfcc14 <dbl>, mfcc15 <dbl>,
## #   mfcc16 <dbl>, mfcc17 <dbl>, mfcc18 <dbl>, mfcc19 <dbl>, mfcc20 <dbl>,
## #   label <fct>
\end{verbatim}

\hypertarget{multinomial-classification}{%
\subsection{\texorpdfstring{\textbf{Multinomial
Classification}}{Multinomial Classification}}\label{multinomial-classification}}

In the following cells I will perform 10-fold CV 10 times, on the KNN
algorithm classification for different values of K (number of nearest
neighbors to consider during algorithm), and then plot to determine the
best value of K.

\includegraphics{Final-Project-Mkd_files/figure-latex/KNN-CV-1.pdf}

Based on the plot from this validation method, for reducing overall
error (across all classification classes), the optimal \textbf{K= 5}.
Next, I will run 20 iterations of KNN using the validation set approach
(60\% training), this time recording and plotting the error of each
respective genre

Errors:
\includegraphics{Final-Project-Mkd_files/figure-latex/Errors Plot -1.pdf}

\hypertarget{decision-trees-and-extenstions}{%
\subsection{Decision Trees and
Extenstions}\label{decision-trees-and-extenstions}}

In the next section, I will attempt to classify the data using decision
tree algorithms, and various extensions such as pruning the tree,
bagging, and random forests

\hypertarget{pruned-decision-tree}{%
\subsubsection{Pruned Decision Tree}\label{pruned-decision-tree}}

In the following cell I use a single decision tree (pruned via 10 fold
cv for best k parameter) to perform multinomial classification and plot
the classification errors across genres

\hypertarget{tree-summary}{%
\subparagraph{Tree Summary}\label{tree-summary}}

\begin{verbatim}
## 
## Classification tree:
## snip.tree(tree = tree.music, nodes = c(30L, 13L))
## Variables actually used in tree construction:
## [1] "chroma_stft"        "rmse"               "spectral_bandwidth"
## [4] "mfcc17"             "mfcc4"              "mfcc1"             
## [7] "mfcc13"             "mfcc15"             "mfcc5"             
## Number of terminal nodes:  16 
## Residual mean deviance:  2.624 = 1795 / 684 
## Misclassification error rate: 0.48 = 336 / 700
\end{verbatim}

\hypertarget{multinomial-confusion-matrix}{%
\subparagraph{Multinomial Confusion
Matrix}\label{multinomial-confusion-matrix}}

\begin{verbatim}
##            
## tree.pred   blues classical country disco hiphop jazz metal pop reggae rock
##   blues        10         0       5     0      0    1     1   2      0    5
##   classical     0        27       1     1      0    5     0   0      1    0
##   country       4         0       7     2      0    3     1   2      4    9
##   disco         1         0       0     5      2    0     0   4      1    1
##   hiphop        1         1       0     7     17    0     8   1      5    3
##   jazz          4         2       4     0      0   17     0   4      0    5
##   metal         1         0       0     5      3    0    17   0      0    2
##   pop           0         0       2     7      4    1     0  21      0    0
##   reggae        5         1       3     1      4    2     0   2     12    3
##   rock          3         0       2     6      1    0     2   0      3    3
\end{verbatim}

\includegraphics{Final-Project-Mkd_files/figure-latex/Plotting Tree Multinomial Errors-1.pdf}

\hypertarget{randomforest-algorithm}{%
\subsubsection{RandomForest Algorithm}\label{randomforest-algorithm}}

In the next cells I will construct RandomForest models for both
multinomial and binary classification. Note the model construction is
based of running 10-Fold, 10 iteration Cross Validation. Furthermore the
mtry values are selected by running 100 iterations of 10-fold cv and
selecting the mtry that most frequently had the lowest cv error. In this
case, optimal mtry = 7

\includegraphics{Final-Project-Mkd_files/figure-latex/Random Forest-1.pdf}

\hypertarget{variable-importance}{%
\paragraph{Variable Importance}\label{variable-importance}}

\includegraphics{Final-Project-Mkd_files/figure-latex/Variable importance graph-1.pdf}
In the next cell, I have selected the top two important variables and
plot a few genres to see how the points are clustered in a 2D fashion
for visual confirmation purposes

\includegraphics{Final-Project-Mkd_files/figure-latex/Clustering Scatter-1.pdf}
\includegraphics{Final-Project-Mkd_files/figure-latex/Clustering Scatter-2.pdf}

These plots confirm our findings with the variable importance as in the
plot with the 2 most important variables, there are very clear linear
boundaries, but in the plot with the 2 least important variables, there
are no clear clusters.

\hypertarget{multinomial-classification-results}{%
\subsubsection{Multinomial Classification
Results}\label{multinomial-classification-results}}

\includegraphics{Final-Project-Mkd_files/figure-latex/Grouped Barplot Multinomial-1.pdf}

\hypertarget{binomial-classification}{%
\subsection{\texorpdfstring{\textbf{Binomial
Classification}}{Binomial Classification}}\label{binomial-classification}}

\hypertarget{k-nearest-neighbor-algorithm}{%
\subsection{K-Nearest-Neighbor
Algorithm}\label{k-nearest-neighbor-algorithm}}

Next, I will use the KNN algorithm, but as a binary classifier, i.e.~for
each genre I will update the dataframe and change every other genre to
``other'' so that the column as a factor only has 2 levels (said genre,
and other). The goal of this is to improve the accuracy of each genre
without multinomial classification. Furthermore, I will use 10 fold CV
10 times f choose a seperate optimal K value for each genre; providing
those values of K and plotting the errors.

Another assumption, I randomly select 100 ``other'' data points to keep
the dataset balanced, as we will only have 100 points for each
individual genre

\includegraphics{Final-Project-Mkd_files/figure-latex/Plotting Error Values-1.pdf}

\hypertarget{pruned-decision-trees}{%
\subsection{Pruned Decision Trees}\label{pruned-decision-trees}}

In the next cell I will perform the same process, but only using
sub-setted data for binary classification again (200 rows) pruning each
tree by their best k parameter using cross-validation

\includegraphics{Final-Project-Mkd_files/figure-latex/Plotting Tree Binomial Errors-1.pdf}

\hypertarget{randomforest}{%
\subsection{RandomForest}\label{randomforest}}

In this next cell, I will perform Binary classification, in the same
ways as above, using the RandomForest algorithm and the optimal tuned
mtry (which was found to be 7 using a type of cross validation). The
error values depicted in the bar-chart values were found using 10 fold,
10-iteration cross validation.

\includegraphics{Final-Project-Mkd_files/figure-latex/Binary Classification RandomForest-1.pdf}

Variable Importance for Rock and Classical Music (Two very different
sounding genres)

\includegraphics[width=0.5\linewidth]{Final-Project-Mkd_files/figure-latex/Variable importance singel genres-1}
\includegraphics[width=0.5\linewidth]{Final-Project-Mkd_files/figure-latex/Variable importance singel genres-2}

\hypertarget{results}{%
\subsubsection{Results}\label{results}}

\includegraphics{Final-Project-Mkd_files/figure-latex/Gropued barplot-1.pdf}

\end{document}
